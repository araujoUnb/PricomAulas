\documentclass[10pt,hyperref={pdfpagemode=FullScreen},aspectratio=169]{beamer}

\usetheme[progressbar=frametitle]{metropolis}
\usepackage[brazil]{babel}
\usepackage{appendixnumberbeamer}
\usepackage{lipsum}
\usepackage{amsmath}
\usepackage{amssymb}
\usepackage{booktabs}
\usepackage{siunitx}
\usepackage[scale=2]{ccicons}
\usepackage{pgfplots}
\usepgfplotslibrary{dateplot}
\usepackage{tikz}
\usetikzlibrary{positioning, shapes.geometric}
\pgfplotsset{compat=newest} % Allows to place the legend below plot
\usepgfplotslibrary{units} % Allows to enter the units nicely
\usepackage{circuitikz}
\usetikzlibrary{shapes,arrows}
\usepackage{dirtytalk}
\usepackage{xspace}




% Bibliography
\usepackage[
    backend=biber,
    style=ieee,
    natbib=true,
    url=false, 
    doi=true,
    eprint=false
]{biblatex}
%\addbibresource{references.bib}


\newcommand{\themename}{\textbf{\textsc{metropolis}}\xspace}
\newcommand{\universidade}{Universidade de Brasília}
\newcommand{\doctitle}{Princípios de Comunicação para Engenharia}

\definecolor{mpigreen}{HTML}{007977}
\setbeamercolor{frametitle}{bg=mpigreen}

\title{\doctitle}

\author{Daniel Araújo}
\institute{\universidade}
\titlegraphic{\hfill\includegraphics[height=1.5cm]{../logo}}

\title{Ementa da Disciplina}

\author{Prof. Daniel Costa Araújo}

\usepackage{graphicx}
\usepackage{subfigure}
\usepackage{verbatim}
\begin{document}


\frame{\titlepage}


\begin{frame}
  \frametitle{Conteúdo}

  \begin{itemize}
    \item Modulações lineares
    \item Modulação Angular
    \item Desempenho de receptores
    \item Comunicação digital em banda base
  \end{itemize}

\end{frame}

\begin{frame}
  \frametitle{Programa}
\begin{columns}[T]
  \begin{column}{0.5\textwidth}
    \begin{block}{1.Espaço de Sinais}
      \begin{itemize}
        \item Definições de Sinais e Sistemas
        \item Sistemas de banda-básica e banda-passante
        \item Processos aleatórios
      \end{itemize}
    \end{block}
  
    \begin{block}{2. Modulações lineares}
      \begin{itemize}
        \item AM-DSB
        \item AM-convencional
        \item SSB
        \item VSB
        \item Modulação em canais com ruído Gaussiano
      \end{itemize}
    \end{block}
  \end{column}

  \begin{column}{0.5\textwidth}
    \begin{block}{3. Modulação Angular}
      
       \begin{itemize}
        \item Modulação PM
        \item Modulação FM
        \item Arquiteturas de transmissor e receptor
        \item Análise em canais com ruído Gaussiano
       \end{itemize}
  
    \end{block}
  
    \begin{block}{4. Comunicação digital em banda base }
      \begin{itemize}
        \item Amostragem e Quantização
        \item PCM linear e não-linear
        \item Pulso formatador
        \item critérios de Nyquist para redução de ISI; 
        \item relação entre taxa de transmissão e largura de banda
      \end{itemize}
    \end{block}
  \end{column}
\end{columns}
  
\end{frame}



\begin{frame}
  \frametitle{Avaliações}

  A avaliação da disciplina será feita a partir da média ponderada das avaliações teóricas $A1$, $A2$, $A3$ e as práticas $L1$, $L2$ e $L3$.
  

  \begin{equation*}
      NP =  \frac{A1 + A2 + 2*A3}{4} \hspace{1cm} NL =  \frac{L1 + L2 + 2*L3}{4}
  \end{equation*}
  

  \begin{block}{Condições para a aprovação}
    \begin{itemize}
      \item É obrigatório a participação na última prova e na última atividade laboratório do semestre. Portanto, caso o aluno não esteja presente ou tire zero, ele será automaticamente reprovado!
      \item $ NP \geq 5 $ e $ NL \geq 5 $
      \item Garantir a presencialidade de pelo menos 75\% nas aulas
    \end{itemize}
  Em atendendo  aos requesitos de aprovação a nota final será
  $$
  NF = 0.2*NL + 0.8*NP
  $$
  \end{block}
  
\end{frame}


\begin{frame}
  \frametitle{Referências}

[1] S. Haykin e M. Moher, Sistemas modernos de comunicações wireless. Bookman, 2009, isbn: 9788577801558.

[2] B. Lathi e Z. Ding, Modern Digital and Analog Communication Systems, sér. Oxf Ser Elec Series. Oxford University Press, 2010, isbn: 9780195384932.

[3] J. Rochol, Sistemas de Comunicação sem Fio: Conceitos e Aplicações, sér. Série Livros Didáticos Informática UFRGS. Bookman Editora, isbn: 9788582604564.

[4] V. SOARES NETO, Sistemas de comunicação serviços, modulação e meios de transmissão. São Paulo Erica, 2015, isbn: 9788536522098.

[5] P. H. YOUNG, Técnicas de Comunicação Eletrônica, 5. Editora Pearson, isbn: 9788576050490.

[6] J. Proakis e M. Salehi, Fundamentals of Communication Systems. Pearson Education, 2013, isbn: 9780133354942.

\end{frame}
\end{document}