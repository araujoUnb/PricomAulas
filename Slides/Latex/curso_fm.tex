\documentclass{beamer}
\usepackage{amsmath}

\title{Modulação FM}
\author{Prof. Daniel Costa Araújo}
\date{\today}
\frametitle{Modulação FM}
\usepackage{tikz}



\usepackage{graphicx}
\logo{\includegraphics[height=1cm]{../../Figs/UNBS-300x150.png}}
\setbeamertemplate{headline}{%
  \leavevmode%
  \hbox{%
    \begin{beamercolorbox}[wd=\paperwidth,ht=2.5ex,dp=1.125ex]{palette tertiary}%
      \usebeamerfont{title in headline}{\hspace*{2em}\insertlogo\hspace*{2em}}
    \end{beamercolorbox}}}



\begin{document}

\begin{frame}
    \frametitle{Conceito}

    Em uma definição geral, a frequência é taxa de variação com que a fase do sinal varia ao longo do tempo

$$
f_i(t) = f_c + \frac{1}{2\pi}\frac{d}{dt} \phi (t)
$$

\end{frame}


\end{document}