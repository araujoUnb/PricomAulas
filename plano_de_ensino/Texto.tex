  \begin{snugshade}
  \section{Descrição Geral}
  \end{snugshade}
  
 Esta disciplina irá cobrir os fundamentos de comunicações analógico e digital. Será estudado as técnicas de transmissão analógica que inclui modulação em amplitude, modulação angular, e o impacto do ruído em sistemas analógicos. O estudo de sistemas digitais irá compreender aspectos de comunicação em banda-base como taxa de amostragem, de transmissão e de símbolos; codificação de linha e formatação de pulsos: análise espectral de codificações comumente usadas; interferência entre símbolo (ISIS) e critérios de Nyquist para redução de ISI; relação entre taxa de transmissão e largura de banda. 
  

  
  \begin{snugshade}
  \section{Pré-requisitos} 
  \end{snugshade}
  
  \begin{itemize}
      \item FGA 201642 Métodos Matemáticos Engenharia OU \item FGA 120952 Sinais e Sistemas Para Engenharia 
      \item FGA 118991 Teoria de Circuitos Eletrônicos 1
  \end{itemize}
    
  \begin{snugshade}
  \section{Professor}
  \end{snugshade}
  Nome: \professores

   O contato poder ser via  \href{mailto:daniel.araujo@unb.br}{e-mail (daniel.araujo@unb.br)}, através do Microsoft Teams, ou SIGAA. 


  \begin{snugshade}
  \section{As aulas} % estratégias relevantes adotadas para alcançar os objetivos.
  \end{snugshade}

    O curso será ministrado presencialmente no campus da FGA. O material da disciplina será disponibilizado através do SIGAA 

    As aulas práticas serão executadas pelos alunos através de plataforma de código aberto seguindo os roteiros disponibilizados no sistema.  
    
  
  \begin{snugshade}
  \section{Ementa} 
  \end{snugshade}
  
  \begin{enumerate}
      \item Modulações lineares
      \item Modulação Angular
      \item Ruído em sistemas analógicos 
      \item Comunicação digital em banda base 
  \end{enumerate}
  
\newpage
  \begin{snugshade}
  \section{Programa} 
  \end{snugshade}
  
  \begin{enumerate}
      \item Modulações lineares
      \begin{itemize}
          \item AM-DSB
          \item AM-convencional
          \item SSB
          \item VSB
          \item PLL
      \end{itemize}
      \item Modulação Angular
        \begin{itemize}
          \item Modulação PM
          \item Modulação FM
          \item Métodos de geração direto e indireto (Armstrong)
      \end{itemize}
      \item Ruído em sistemas analógicos 
      \begin{itemize}
          \item Definição e propriedades básicas de ruído branco gaussiano
          \item Definição de relação sinal-ruído (SNR)
          \item Expressões para SNR à entrada e saída de demoduladores básicos
      \end{itemize}
      \item Comunicação digital em banda base 
      \begin{itemize}
          \item Amostragem e Quantização
          \item PCM linear e não-linear
          \item Pulso formatador
          \item interferência entre símbolo (ISIS)
          \item critérios de Nyquist para redução de ISI; \item relação entre taxa de transmissão e largura de banda;
      \end{itemize}
  \end{enumerate}
  
  \begin{snugshade}
  \section{Avaliação}
  \end{snugshade}
  
  A avaliação da disciplina será feita a partir da média ponderada das avaliações teóricas $A1$, $A2$, $A3$ e as práticas $L1$, $L2$ e $L3$.
  

  \begin{equation*}
      NP =  \frac{A1 + A2 + 2*A3}{4} \hspace{1cm} NL =  \frac{L1 + L2 + 2*L3}{4}
  \end{equation*}
  
  Para a aprovação do aluno, as seguintes condições devem ser atendidas:
    \begin{itemize}
      \item É obrigatório a participação na última prova e na última atividade laboratório do semestre. Portanto, caso o aluno não esteja presente ou tire zero, ele será automaticamente reprovado;
      \item $ NP \geq 5 $ e $ NL \geq 5 $;
      \item Garantir a presencialidade de pelo menos 75\% nas aulas.
    \end{itemize}
  Em atendendo  aos requesitos de aprovação a nota final será
  $$
  NF = \frac{0.8*NL + 0.2*NP}{2}.
  $$
 
  
   \newpage 
  \begin{snugshade}
    \section{Referências}
\end{snugshade}

[1] S. Haykin e M. Moher, Sistemas modernos de comunicações wireless. Bookman, 2009, isbn: 9788577801558.


[2] B. Lathi e Z. Ding, Modern Digital and Analog Communication Systems, sér. Oxf Ser Elec Series. Oxford University Press, 2010, isbn: 9780195384932.


[3] J. Rochol, Sistemas de Comunicação sem Fio: Conceitos e Aplicações, sér. Série Livros Didáticos Informática UFRGS. Bookman Editora, isbn: 9788582604564.


[4] V. SOARES NETO, Sistemas de comunicação serviços, modulação e meios de transmissão. São Paulo Erica, 2015, isbn: 9788536522098.


[5] P. H. YOUNG, Técnicas de Comunicação Eletrônica, 5. Editora Pearson, isbn: 9788576050490.


[6] J. Proakis e M. Salehi, Fundamentals of Communication Systems. Pearson Education, 2013, isbn: 9780133354942.